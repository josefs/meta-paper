%-----------------------------------------------------------------------------
%
%               Template for sigplanconf LaTeX Class
%
% Name:         sigplanconf-template.tex
%
% Purpose:      A template for sigplanconf.cls, which is a LaTeX 2e class
%               file for SIGPLAN conference proceedings.
%
% Guide:        Refer to "Author's Guide to the ACM SIGPLAN Class,"
%               sigplanconf-guide.pdf
%
% Author:       Paul C. Anagnostopoulos
%               Windfall Software
%               978 371-2316
%               paul@windfall.com
%
% Created:      15 February 2005
%
%-----------------------------------------------------------------------------


\documentclass[preprint]{sigplanconf}

% The following \documentclass options may be useful:
%
% 10pt          To set in 10-point type instead of 9-point.
% 11pt          To set in 11-point type instead of 9-point.
% authoryear    To obtain author/year citation style instead of numeric.

\usepackage{amsmath}
\usepackage{array}

\usepackage[unicode=true]{hyperref}
\usepackage{color}
$if(lhs)$
\lstnewenvironment{code}{\lstset{language=Haskell,basicstyle=\small\ttfamily}}{}
$endif$
$if(highlighting-macros)$
$highlighting-macros$
$endif$

\newcommand{\TODO}[1]{\(\spadesuit\){\bf TODO:} {\bf \color{red} #1}\\}

\begin{document}

\conferenceinfo{WXYZ '05}{date, City.} 
\copyrightyear{2005} 
\copyrightdata{[to be supplied]} 

\titlebanner{banner above paper title}        % These are ignored unless
\preprintfooter{short description of paper}   % 'preprint' option specified.

\title{$title$}
%\subtitle{Subtitle Text, if any}

$for(author)$
\authorinfo{$author$}{}{}
$endfor$
%% \authorinfo{Name1}
%%            {Affiliation1}
%%            {Email1}
%% \authorinfo{Name2\and Name3}
%%            {Affiliation2/3}
%%            {Email2/3}

\maketitle

\begin{abstract}
This paper argues for a new methodology for writing high performance
Haskell programs by using Embedded Domain Specific Languages. 

We exemplify the methodology by describing a complete library,
meta-repa, which is a reimplementation of parts of the repa
library. The paper describes the implementation of meta-repa and
contrasts it with the standard approach to writing high performance
libraries. We conclude that even though the embedded language approach
has an initial cost of defining the language and some syntactic
overhead it gives a more natural programming model, stronger
performance guarantees, better control over optimizations, simpler
implementation of fusion and inlining and allows for moving type level
programming down to value level programming in some cases. We also
provide benchmarks showing that meta-repa is as fast, or faster, than
repa.

Furthermore, meta-repa also includes push arrays and we demonstrate
their usefulness for writing certain high performance kernels such as
FFT.
\end{abstract}

\category{CR-number}{subcategory}{third-level}

\terms
term1, term2

\keywords
keyword1, keyword2

$body$
%% \section{Introduction}

%% The text of the paper begins here.

\appendix
\section{Appendix Title}

This is the text of the appendix, if you need one.

\acks

Acknowledgments, if needed.

% We recommend abbrvnat bibliography style.

\bibliographystyle{abbrvnat}

% The bibliography should be embedded for final submission.

\begin{thebibliography}{}
\softraggedright

\bibitem[Smith et~al.(2009)Smith, Jones]{smith02}
P. Q. Smith, and X. Y. Jones. ...reference text...

\end{thebibliography}

\end{document}
